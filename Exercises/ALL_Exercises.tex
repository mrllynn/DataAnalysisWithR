
% -----------------------------------------------------------------------------
% IMPLEMENTING THE DOCUMENT AND ITS OPTIONS
% -----------------------------------------------------------------------------

\documentclass[11pt]{article}\usepackage[]{graphicx}\usepackage[]{color}
%% maxwidth is the original width if it is less than linewidth
%% otherwise use linewidth (to make sure the graphics do not exceed the margin)
\makeatletter
\def\maxwidth{ %
  \ifdim\Gin@nat@width>\linewidth
    \linewidth
  \else
    \Gin@nat@width
  \fi
}
\makeatother

\definecolor{fgcolor}{rgb}{0.345, 0.345, 0.345}
\newcommand{\hlnum}[1]{\textcolor[rgb]{0.686,0.059,0.569}{#1}}%
\newcommand{\hlstr}[1]{\textcolor[rgb]{0.192,0.494,0.8}{#1}}%
\newcommand{\hlcom}[1]{\textcolor[rgb]{0.678,0.584,0.686}{\textit{#1}}}%
\newcommand{\hlopt}[1]{\textcolor[rgb]{0,0,0}{#1}}%
\newcommand{\hlstd}[1]{\textcolor[rgb]{0.345,0.345,0.345}{#1}}%
\newcommand{\hlkwa}[1]{\textcolor[rgb]{0.161,0.373,0.58}{\textbf{#1}}}%
\newcommand{\hlkwb}[1]{\textcolor[rgb]{0.69,0.353,0.396}{#1}}%
\newcommand{\hlkwc}[1]{\textcolor[rgb]{0.333,0.667,0.333}{#1}}%
\newcommand{\hlkwd}[1]{\textcolor[rgb]{0.737,0.353,0.396}{\textbf{#1}}}%
\let\hlipl\hlkwb

\usepackage{framed}
\makeatletter
\newenvironment{kframe}{%
 \def\at@end@of@kframe{}%
 \ifinner\ifhmode%
  \def\at@end@of@kframe{\end{minipage}}%
  \begin{minipage}{\columnwidth}%
 \fi\fi%
 \def\FrameCommand##1{\hskip\@totalleftmargin \hskip-\fboxsep
 \colorbox{shadecolor}{##1}\hskip-\fboxsep
     % There is no \\@totalrightmargin, so:
     \hskip-\linewidth \hskip-\@totalleftmargin \hskip\columnwidth}%
 \MakeFramed {\advance\hsize-\width
   \@totalleftmargin\z@ \linewidth\hsize
   \@setminipage}}%
 {\par\unskip\endMakeFramed%
 \at@end@of@kframe}
\makeatother

\definecolor{shadecolor}{rgb}{.97, .97, .97}
\definecolor{messagecolor}{rgb}{0, 0, 0}
\definecolor{warningcolor}{rgb}{1, 0, 1}
\definecolor{errorcolor}{rgb}{1, 0, 0}
\newenvironment{knitrout}{}{} % an empty environment to be redefined in TeX

\usepackage{alltt}
\usepackage[left=21mm, right=18mm, top=20mm, bottom=18mm]{geometry}
\linespread{1.4}

% *****************************************************************************
% ADD HERE THE TOPIC OF THE LECTURE!
\newcommand{\exercisetopic}{ALL EXERCISES}
% *****************************************************************************

% Add information about the title section of the file
\author{Sonja Hartnack, Terence Odoch \& Muriel Buri}
\date{\today}
\title{\vspace{-3ex}Practical Exercises for \textbf{\exercisetopic}}
\date{October 2017}

\usepackage[round,sectionbib]{natbib} \bibliographystyle{ims}
\usepackage{graphicx}
\usepackage{float}
\usepackage{url}
\usepackage{color}
\usepackage{amsmath, amssymb}
\usepackage{colortbl, xcolor} % enable colored rows in table
\usepackage{color} % enable colored rows in table
\usepackage{bibentry} % Include Full BibTeX Entry Inside Slides
\usepackage{nicefrac}
\usepackage{soul}

\graphicspath{{figures/}}

% Adjusting font type
\usepackage[sfdefault,lf]{carlito}
\usepackage[T1]{fontenc}
\renewcommand*\oldstylenums[1]{\carlitoOsF #1}

% Add footer and header to the file
\usepackage{lastpage}
\usepackage{fancyhdr}
\renewcommand{\headrulewidth}{0.4pt}% Default \headrulewidth is 0.4pt
\renewcommand{\footrulewidth}{0.4pt}% Default \footrulewidth is 0pt
\pagestyle{fancy}
\lhead{Data Analysis with R: Exercises} % left header
\rhead{\exercisetopic} % right header
\lfoot{Sonja Hartnack, Terence Odoch \& Muriel Buri} % left footer
\rfoot{\thepage\ of \pageref{LastPage}} % right footer
\cfoot{} % get rid of the centered page number

% -----------------------------------------------------------------------------
% BEGINNING OF ACTUAL DOCUMENT
% -----------------------------------------------------------------------------
\IfFileExists{upquote.sty}{\usepackage{upquote}}{}
\begin{document}

\maketitle
\thispagestyle{fancy}

\section*{Exercise 1}
\begin{itemize}
\item Open R Studio
\item Open a new R-Script
\item Load data set \texttt{chickwts}

%
\item Do summary statistic (numerically and graphically)

Anova, lm, which groups differ, Bonferroni, Tukey-Anscombe
Histogram with density line
Normally distributed weights
\end{itemize}
%
\section*{Exercise 2}
\begin{itemize}
\item Create a data frame with 3 columns.

\end{itemize}
%
\section*{Exercise 3}
\begin{itemize}
\item Install package \texttt{MASS}.

%
\item Load data set \texttt{bacteria}.

%
\item Do summary statistic (numerically and graphically).

%
\item Select only observations collected during the second week.

\end{itemize}
%
\section*{Exercise 4}
What is conceptionally the difference between these bracket types ([...], (...))?
\begin{knitrout}\scriptsize
\definecolor{shadecolor}{rgb}{0.969, 0.969, 0.969}\color{fgcolor}\begin{kframe}
\begin{alltt}
\hlstd{chickwts[,} \hlnum{2}\hlstd{]}
\hlkwd{summary}\hlstd{(}\hlkwd{aov}\hlstd{(weight} \hlopt{~} \hlstd{feed,} \hlkwc{data} \hlstd{= chickwts))}
\end{alltt}
\end{kframe}
\end{knitrout}
%

%
\section*{Exercise 5}
\begin{itemize}
\item How many levels has the factor variable \texttt{trt} from \texttt{bacteria}?

\item Define a new variable \texttt{trt.new} in which you combine the levels
\texttt{drug} and \texttt{drug+} into one single level and label it as \texttt{treated}.
The new variable \texttt{trt.new} should in the end have two levels: \texttt{placebo} and \texttt{treated}.

\item Do summary statistics for \texttt{placebo} and \texttt{treated} group.

\end{itemize}
%
\section*{Exercise 6}
\begin{itemize}
\item Load data set \texttt{ToothGrowth}.

\item Do summary statistic (numerically and graphically).

\item Define additional column \texttt{dose.fac} by converting the numeric variable \texttt{dose} into a factor variable.

%
\item Are the tooth length measurements normally distributed within the treatment
(\texttt{supp}: VC or OJ) and within in the different doses (\texttt{dose}: 0.5, 1, 2)?

\end{itemize}
%
\section*{Exercise 7}
\begin{itemize}
\item Import the data set \texttt{perulung\_ems.csv} (taken from Kirkwood and
Sterne, 2nd edition) into R. \newline
Data from a study of lung function among children living in a deprived suburb of
Lima, Peru. \newline
Variables:
\begin{itemize}
\item \texttt{fev1}:  in liter, ''Forced Expiratory Volume in 1 second'' measured
by a spirometer. This is the maximum volume of air which the children could breath
out in 1 second
\item \texttt{age}: in years
\item \texttt{height}: in cm
\item \texttt{sex}: 0 = girl, 1 = boy
\item \texttt{respsymp}: respiratory symptoms experienced by the child over the
previous 12 months
\end{itemize}
\item What \textit{delimiter} do you need to choose?

\item Do all variables have the correct data type (numeric, integer, factor)?
If not, do correct and / or define them.

\end{itemize}
%

%
Check for heteroscedascity or homogeneity of variances

%
\section*{Exercise 8}
Apply the summary statistics to the \texttt{perulung\_ems} and \texttt{ToothGrowth}
data set.

%
\section*{Exercise 9A: Plausibility Checks}
\begin{itemize}
\item What can go wrong?
\item Identify different strategies for spotting these potential errors.
\begin{itemize}
\item Logical errors
\item Spelling mistakes
\end{itemize}
\item Import the data set \texttt{bacteria\_plausibility\_check.csv} to R.

\item Detect the \textbf{six} errors in the imported data set
\texttt{bacteria\_plausibility\_check.csv} in R.

\item Find possible solutions in R how to handle these challenges.

\item Do all variables have the correct data type (numeric, integer, factor)?
- If not, do correct / define them.

\end{itemize}
%
\section*{Exercise 9B: Missing Values}
\begin{itemize}
\item Check out the difference between the different missing values

%
\item Create a vector with missing values and determine the mean and median

%
\item If \texttt{x = c (22,3,7,NA,NA,67)} what will be the output for the R
statement \texttt{length(x)}?

%
\item If \texttt{x = c(NA,3,14,NA,33,17,NA,41)} which line of R code removes
all occurrences of NA in x.

%
\item If \texttt{y = c(1,3,12,NA,33,7,NA,21)} what R statement will replace
all occurrences of NA with 11?

%
\item If \texttt{x = c(34,33,65,37,89,NA,43,NA,11,NA,23,NA)} then what will
count the number of occurrences of NA in \texttt{x}?

%
\item Create a vector and find the number of missing values and their position

%
\item Now, create the vector x2 and assess the difference to x1

%
\item What is the meaning of "NA" versus "NaN"?
%
\item Replace the missing values in x1 with a 0, and check that no NAs are present
try two different commands to coerce the NAs into 0

\end{itemize}
%
\section*{Exercise 10}
\begin{itemize}
\item Import the data set \texttt{water\_errors.csv} to R:
A data frame with $61$ observations on the following $6$
variables.
\begin{itemize}
\item \textbf{location}: a factor with levels \texttt{North} and \texttt{South} indicating
whether the town is as north as Derby.
\item \textbf{town}: the name of the town.
\item \textbf{mortality}: averaged annual mortality per 100.000 male inhabitants.
\item \textbf{hardness}: calcium concentration (in parts per million).
\item \textbf{smoker}: If there are any smokers living in town.
\item \textbf{num.of.cig}: In case, smokers live in town, what number of
cigarettes do they smoke per day.
\end{itemize}

%
%
%
%
%
\item Detect the errors in the imported data set \texttt{water\_errors.csv} in R.

\item Find possible solutions in R how to handle these challenges.

\item Do all variables have the correct data type (numeric, integer, factor)?
- If not, do correct / define them.

\end{itemize}
%
\section*{Exercise 11}
\begin{itemize}
\item Apply the two-sided two sample t-test to suitable variables of the data set \texttt{ToothGrowth}.

\item Interpret the results.

\item Apply the two-sided t-test to the \texttt{perulang\_ems} data set

\end{itemize}
%
%
\section*{Exercise 12}
\begin{itemize}
\item Apply the Chi-square Test and the fisher exact test to the whole
\texttt{bacteria} data set.

\item Apply the Chi-square Test and the fisher exact test to the subset of
\texttt{bacteria} containing only the observations taken in week 2.
Are there any issues?

\item Repeat this exercise by using the (previously defined) combined
\texttt{trt.new} variable with the two levels \texttt{treated} and \texttt{drug}.

\item Could you also obtain the odds ratios?

\item Try also a logistic regression in R. Ask Google for help!

\end{itemize}
%
\section*{Exercise 13A: Outside plot frame}
\begin{itemize}
\item Type \texttt{demo(graphics)} in your console and press enter.
This command shows you a nice demonstration of possible R graphics.

\item Change the x-axis and y-axis labelling of a boxplot plotting the
\texttt{len} variable of the \texttt{ToothGrowth} data set.

\item How do you set a main title for your above plot?

\item What does the following command do?


%
\item We have six different feed types in \texttt{chickwts}. Try to plot two
separate boxplots for \texttt{casein} and \texttt{horsebean} and set the same
minimum and maximum for the y-axis. Use the function \texttt{subset} for doing so.


\item How do you enlarge the font size of the axis as well as the axis labels
of the following plot with the \texttt{perulung} data set?


\item Label the x-axis of the following plot with ''Vitamin C in $\mu$g''. Use the
greek letter for $\mu$.


\item Read \url{http://www.statmethods.net/advgraphs/parameters.html}.
\end{itemize}

\section*{Exercise 13B: Inside the square of the plot}
\begin{itemize}
\item Type \texttt{demo(graphics)} in your console and press enter.
\item Add a legend to the following barplot. Are there several different solutions
for this?


\item Add a density line to this histogram.


\item Add a \textbf{dotted red} linear regression line to the following plot.


\item Color the points in the following plot according to the \texttt{sex} variable.


\item Add two linear regression lines separately for \texttt{female} and \texttt{male}to the following plot.


\item Color the points in the following plot according to the \texttt{supp} variable.
Use different point characters (\texttt{pch}) based on the \texttt{supp} variable.


\item Read \url{http://www.statmethods.net/advgraphs/parameters.html}.
\end{itemize}

\section*{Exercise 14}
\begin{itemize}
\item Load the below data set and for further information check the command
\texttt{?water}.

\item Try to plot the variables \texttt{mortality} against \texttt{hardness} from
the \texttt{water} data set.

\item Add a main title to the above plot (\texttt{mortality} against
\texttt{hardness}).

\item Change the ...
\begin{enumerate}
\item font size of the axis annotation
\item font size of the x- and y-axis labels
\item the point sizes within the plot
\end{enumerate}
... of the above plot (\texttt{mortality} against \texttt{hardness}).

\item Looking at the above plot: Do you think the two variables \texttt{hardness}
and \texttt{mortality} correlate? What function do you use to find out the correlation
coefficient? Do they have a positive or a negative correlation coefficient? How
do you interpret the correlation coefficient in your own words?

\item In the \texttt{water} data set, can you graphically find out if there is a
difference between the the two variables \texttt{hardness} and \texttt{mortality}
conditional on the \texttt{location} (\texttt{North}, \texttt{South}).

\item Add a legend to the above plot so that you can easily differentiate the
locations (\texttt{North} or \texttt{South}) of the observations.

\item Do a barplot of the variable \texttt{location} from the \texttt{water} data
set.

\item ADDITIONAL: Try if any of these following plotting functions can be
applied to the data sets \texttt{perulung} or \texttt{ToothGrowth}.


\end{itemize}

\section*{Exercise 15}
\begin{itemize}
\item Download the .R file \texttt{ANOVA\_with\_chickwts.R} from the switch drive
and have another look on how we applied the anova to the \texttt{chickwts}
data set.
\item Load the \texttt{ToothGrowth} data set into R and encode the numeric variable
\texttt{dose} as a factor variable. Define the new factor variable as
\texttt{dose.fac} with the three levels \texttt{low}, \texttt{med} and
\texttt{high} and add it to the data frame of \texttt{ToothGrowth}.

\item Visualize the variable \texttt{len} per \texttt{dose} level in a boxplot.

\item With the help of the R-commands written in the
\texttt{ANOVA\_with\_chickwts.R}  file, apply a analysis of variance (ANOVA) to
the data set \texttt{ToothGrowth}

\end{itemize}

\section*{Exercise 16}
\begin{itemize}
\item Download the .R file \texttt{LM\_with\_water.R} from the switch drive
and have another look on how we applied the linear model to the \texttt{water}
data set.
\item Reuse these commands to fit a simple as well as multiple linear regression
model to the data set of \texttt{perulung\_ems}. Use \texttt{fev1} as your
response variable $y$.

\end{itemize}

\section*{Exercise 17}
\begin{itemize}
\item Load the \texttt{ToothGrowth} data set and run the following four
linear regression models.



\item Have a look at the summary of these models.

\item How do you interpret the model coefficients?
\item Which model is best?

\end{itemize}

\section*{Exercise 18}
\begin{itemize}
\item Load the \texttt{water} data set and fit a multiple linear regression model.
Use \texttt{mortality} as your response variable and add \texttt{hardness} and
\texttt{location} as an explanatory variable.

\item Check the underlying model assumptions.

\item Add an interaction term between \texttt{hardness} and \texttt{location}
to the above estimated multiple linear regression model.

\item Interpret the interaction coefficient \texttt{hardness:locationSouth}.
\item Check the underlying model assumptions.

\item Which one is the better model? With or without the interaction term?

\item How to derive confidence intervals for the regression coefficient
of \texttt{hardness} and \texttt{location}?

\end{itemize}


\section*{Exercise 19}
{\bfseries Hypothetical example} - from Kirkwood and Sterne, Medical Statistics, 2nd ed., p. 177
\begin{itemize}
\item Read in the data set \texttt{lepto}. This study presents a serology survey of
leptospira sero-prevalence in rural and urban areas of the west indies.

\item Encode the numeric variable \texttt{antibodies} as a factor with levels $0$ and $1$.

\item Make a crosstable with the risk factor \texttt{exposure} and \texttt{antibodies}.

\item Run a Chi-squared test, a Fisher's exact test and a logistic regression (\texttt{glm})
to assess if the \texttt{exposure} (living in rural vs. urban areas) is a risk factor.

\item Create a subset for \texttt{male} and \texttt{female} based on the variable
\texttt{gender}.

\item Repeat the crosstable, Chi-squared test, Fisher's exact test and a logistic
regression (\texttt{glm}) for the subsets \textbf{separately}.

\item Does the conclusion of your research question change with the analysis of
the subsets? (Research question: Is the \texttt{exposure} (rural and urban areas)
a risk factor?)
\item Fit a logistic regression model (\texttt{glm}) with \texttt{exposure} and
\texttt{gender} as explanatory variables.

\item \textbf{SPECIAL FOR GUMA}: Is \texttt{exposure} being from an urban area
a risk factor?


\end{itemize}

\end{document}
